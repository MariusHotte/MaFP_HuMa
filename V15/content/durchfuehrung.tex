\section{Durchführung}
\label{sec:Durchführung}
\begin{itemize}
\item
Es wird eine Strom-Spannungs-Kennlinine in 10$\,$V Schritten aufgenommen. Mittels dieser
Werte kann die Depletionsspannung abgeschätzt werden, welche für die
weiteren Aufgabenteile von Bedeutung ist.

\item Für eine Untersuchung des Pedestals und Noise wird ein
$Pedestal\,\, Run$ durchgeführt. Bei diesem erfolgt die Aufnahme der Daten ohne externe Quelle. Somit wird das "Offset-Rauschen" für jeden Streifen ermittelt. Mit den aufgenommenen Werten
werden Noise, Pedestals und Common Mode bestimmt und graphisch dargestellt.

\item Es wird eine Kalibrationsmessung durchgeführt um die ADC Counts
in deponierte Energie umzurechnen. Hierfür wird das Auslesesignal in Abhängkeit der deponierten Ladung vermessen. Um eine statistisch fundierte Aussage treffen zu können, wird die Kalibrationsmessung für fünf verschiedene Channel durchgeführt. Durch die Berechnung des Mittelwerts kann dann eine Aussage über den ganzen Sensor getroffen werden. Die Bestimmung des mathematischen Zusammenhangs, sowie die Umrechnung in Energie der Einheit $eV$ wird in Kapitel \ref{nooise} erläutert.


\item
Zur Vermessung des Streifensensors wird ein Laser verwendet. Er wird mit
Hilfe einer Mikrometer-Schraube in $10\mu m$ Schritten über den Sensor gefahren
und es wird jeweils die Signale der einzelnen Channel gemessen, um daraus
die Breite eines Sensorstreifens zu bestimmen.

\item Mit Hilfe des Lasers und unter Verwendung einer $\beta-$-Quelle wird eine Bestimmung der Charge Collection Efficiency in Abhängigkeit der Dicke des Sensors und der Eindringtiefe des Lasers durchgeführt.

\item Zuletzt wird eine Messung von 1000000 Events durchgeführt und das Energiespektrum
des Scans untersucht.
\end{itemize}
