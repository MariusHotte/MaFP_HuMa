\section{Diskussion}
\label{sec:Diskussion}

\subsection{Aktivität}
Der Vergleich des experimentell bestimmten Werts mit dem theoretischen Wert aus der Versuchsanleitung \cite{skript} ergibt eine Abweichung von $\SI{273.6}{\percent}
$.  Die hohe Abweichung zum theoretischen Wert ist nicht offensichtlich, da der Energieverlust bei dem verwendeten Druck und der betrachteten Strecke vernachlässigbar ist(siehe Kapitel \ref{sec:Vorbereitung}).

\subsection{Foliendicke}
Die in dem Kapitel \ref{sec:foliendicke} berechnete Dicke der Folie beträgt $d_{\text{exp}} = \SI{3.5+-0.2}{\micro\meter}
$. Somit weißt sie eine relative Abweichung von $\SI{74+-11}{\percent}
$ zum angegebenen Wert von $\SI{2}{\micro\meter}$ auf. Durch Betrachtung von Abbildung \ref{fig:Druck} wird nicht ersichtlich, woraus die großen Abweichung resultiert, da der mathematisch Zusammenhang, bis auf lokale Schwankungen, sehr gut durch die lineare Ausgleichsrechnung bestimmt werden kann. Die Steigung beider Ausgleichsrechnungen 
\begin{align}
  a_{\text{ohne Folie}} &= \SI{-0.0124+-0.0002}{\volt\per\milli\bar}
 \\
  a_{\text{mit Folie}} &= \SI{-0.0119+-0.0007}{\volt\per\milli\bar}
 \\
\end{align}
sind fast identisch, sodass ein Fehler beim Austausch der Folie ausgeschlossen werden kann.

\subsection{Differentieller Wirkungsquerschnitt}
In Abbildung \ref{fig:Wirkungsquerschnitt} ist deutlich zu erkennen, dass sich das Maximum der experimentell bestimmten Werte nicht bei einem Winkel von $\SI{0}{\degree}$ befindet. Dieser Umstand ist auf eine fehlerhafte Justage des Winkels der Folie zur Blende zurückzuführen. Unter Berücksichtigung dieses Umstands stimmt der theoretische Verlauf mit den experimentellen Wert grob überein. Für große Winkel geht die theoretische Wirkungsquerschnitt aufgrund der Sinus-Funktion gegen Null, welches sich auch in den Messwerten widerspiegelt. Ebenfalls der Verlauf bei $\SI{0}{\degree}$ gegen $\infty$ entspricht dem erreichen eines Maximums bei den experimentell ermittelten Werten.

\subsection{Mehrfachstreuung}
An den Messwerten aus Kapitel \ref{sec:Mehrfachstreuung} ist der Effekt der Mehrfachstreuung gut ersichtlich. Unter Verwendung einer dicken Folie treten mehrfach Wechselwirkung mit den Kernen der Folie auf, welches zu einem größeren Energieverlust als bei einer dünneren Folie führt. Folglich sind unter Verwendung einer dickeren Folien weniger Counts zu erwarten.

\subsection{Z-Abhängigkeit}
Aufgrund der Rutherfordschen Streuformel \ref{wq} wird ein quadratischer Zusammenhang zur Ordnungszahl der Folie erwartet. Dieses Umstand kann durch die Messergebnisse nicht bestätigt werden. Die Messergebnisse zeigen einen, durch die Ausgleichsrechnung verdeutlichten, linearen Zusammenhang zwischen der normierten Intensitäten und der Ordnungszahl der verwendeten Folie. Es ist jedoch zu berücksichtigen, dass die Anzahl der Messwerte begrenzt sind und dadurch die quadratische Abhängigkeit nicht aufgetreten ist.



