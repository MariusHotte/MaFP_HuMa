\section{Durchführung}
\label{sec:Durchführung}
\begin{itemize}
\item Die Vakuumpumpe wird zur Evakuation eingeschaltet und eine Spannung von $U_\text{det}=\SI{12}{\volt}$ am Detektor eingestellt.

\item Die elektronischen Pulse, die durch den Einfall von $\alpha$-Teilchen verursacht werden, werden nach jeder elektronischen Komponente dokumentiert.

\item Um die Foliendicke experimentell zu bestimmen, wird die Pulshöhe in Abhängigkeit des Kammerdrucks mit und ohne Goldfolie gemessen.

\item Zur Bestimmung des differentiellen Wirkungsquerschnitts wird die Zählrate in Abhängigkeit des eingestellten Winkels zwischen Folie und Blende bestimmt.

\item Der Einfluss der Mehrfachstreuung wird untersucht, in dem die Zählrate für zwei verschieden dicke Folien bei gleichem Winkel aufgenommen wird.

\item Zur Bestimmung der Ordungszahlabhängigkeit sind Messwerte zur Verfügung gestellt worden, da die benötigten Folie nicht zur Verfügung standen. 
\end{itemize}

