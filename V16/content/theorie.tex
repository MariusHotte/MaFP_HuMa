\section{Theorie}
\label{sec:Theorie}
\subsection{Zielsetzung}
Ziel des Versuchs ist es, die Streuung von $\alpha$-Teilchen, in Abhängigkeit des Winkels und der Ordnungszahl, an einer dünnen Goldfolie zu untersuchen. 

\subsection{Aktivität}

Die Aktivität $A$ beschreibt die Anzahl der Kernzerfällen eines aktiven Mediums
pro Zeitintervall. Diese nimmt im Laufe der Zeit exponentiell ab und wird durch
die Zerfallskonstante $\lambda$ charakterisiert. Die Aktivität kann wie folgt bestimmt werden:
$$ A(t)=\frac{\text{d}N(t)}{\text{d}t}=A_0\cdot e^{-\lambda t}$$

\subsection{Wechselwirkung mit Materie}

Die Wechselwirkung von $\alpha$-Teilchen mit Materie wird
maßgeblich durch die Wechselwirkung des $\alpha$-Teilchens mit einem
Hüllenelektron oder den Kern gekennzeichnet. Der Energieverlust des Teilchens
kann durch die Bethe-Bloch Formel beschrieben werden.
Die allgemeine Bethe-Bloch Formel

\begin{equation}
    \label{betherel}
    \left<\frac{\text{d}E}{\text{d}x}\right>=\frac{4\pi}{m_e \text{c}^2}\cdot \frac{n z^2ZN}{\beta^2}
    \left(\frac{e^2}{4\pi \epsilon_0}\right)^2\cdot\ln\left(\frac{2 m_e \text{c}^2\beta^2}{I\cdot(1-\beta)}\right)-\beta^2
\end{equation}

gibt den Energieverlust pro Strecke für relativistisch Teilchen an. Wobei $\beta=v/$c, $v$
die momentane Geschwindigkeit des Teilchens, c die Lichtgeschwindigkeit,
$z$ die Ladungszahl des Teilchens, $n$ die Elektronendichte des Teilchens und $I$ die
mittlere Anregungsenergie des Materials beschriebt.\\
Für niedrige Energien der Teilchen, bzw. kleine Teilchengeschwindigkeiten, geht
$\beta=v/c \rightarrow 0$ und die Bethe-Bloch Formel lässt sich reduzieren auf

\begin{equation}
    \label{bethe}
    \left<\frac{\text{d}E}{\text{d}x}\right>=\frac{4\pi n z^2ZN }{m_e v^2}\left(\frac{e^2}{4 \pi \epsilon_0} \right)^2
    \cdot    \ln\left(\frac{2 m_e v^2}{I}\right)\;.
\end{equation}


Die Wechselwirkung des $\alpha$-Teilchens mit einem Hüllenelektron
hat keine großen Winkelabweichungen zufolge und ist nicht messbar. Bei
Wechselwirkungen von $\alpha$-Teilchen den Atomkernen kommt es jedoch zu großen
Winkeländerungen. Die verursachte Streuung lässt sich mittels der Rutherfordschen Streuformel beschreiben. Diese bestimmt den differentiellen Wirkungsquerschnitt in Abhängigkeit der Ordnungszahl $Z$, der Energie der $\alpha$-Teilchen $E_\alpha$ und des Streuwinkels $\theta$. 
\begin{equation}
    \label{wq}
    \frac{\text{d}\sigma}{\text{d}\Omega}(\Theta)
    =\frac{1}{(4\pi \epsilon_0)^2}
    \left( \frac{ z Z e^2}{4E_\alpha}\right)^2 \frac{1}{\sin ^4 \frac{ \Theta} {2}}
\end{equation}













% 2x2 Plot
% \begin{figure*}
%     \centering
%     \begin{subfigure}[b]{0.475\textwidth}
%         \centering
%         \includegraphics[width=\textwidth]{Abbildungen/Schaltung1.pdf}
%         \caption[]%
%         {{\small Schaltung 1.}}
%         \label{fig:Schaltung1}
%     \end{subfigure}
%     \hfill
%     \begin{subfigure}[b]{0.475\textwidth}
%         \centering
%         \includegraphics[width=\textwidth]{Abbildungen/Schaltung2.pdf}
%         \caption[]%
%         {{\small Schaltung 2.}}
%         \label{fig:Schaltung2}
%     \end{subfigure}
%     \vskip\baselineskip
%     \begin{subfigure}[b]{0.475\textwidth}
%         \centering
%         \includegraphics[width=\textwidth]{Abbildungen/Schaltung4.pdf}    % Zahlen vertauscht ... -.-
%         \caption[]%
%         {{\small Schaltung 3.}}
%         \label{fig:Schaltung3}
%     \end{subfigure}
%     \quad
%     \begin{subfigure}[b]{0.475\textwidth}
%         \centering
%         \includegraphics[width=\textwidth]{Abbildungen/Schaltung3.pdf}
%         \caption[]%
%         {{\small Schaltung 4.}}
%         \label{fig:Schaltung4}
%     \end{subfigure}
%     \caption[]
%     {Ersatzschaltbilder der verschiedenen Teilaufgaben.}
%     \label{fig:Schaltungen}
% \end{figure*}
