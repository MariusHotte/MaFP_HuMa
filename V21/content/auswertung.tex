\section{Auswertung}
\label{sec:Auswertung}
\subsection{Bestimmung des lokalen Erdmagnetfeld}
Für die Bestimmung des Erdmagnetfeldes werden die Magnetfeldstärken der
Sweepfeld- und Horizontalfeld in Abhängigkeit der Resonanzfrequenz gemessen.
Die gemessenen Messwerte werden in der Tabelle \ref{tabmess1} dargestellt.
 \begin{table}
   \centering
   \caption{Position der Resonanzstellen für verschiedene Frequenzen}
   \label{tabmess1}
   \sisetup{parse-numbers=false}
   \begin{tabular}{c|c|c|c|c}
     \toprule
    $f$ in Hz & Sweep 1& Horizontal 1 & Sweep 2&Horizontal 2 \\
     \midrule
     100 &  5.66 &  6.84  & 0   &   0   \\
     200 &  5.78 & 7.13  & 0.15 &  0.15  \\
     300 &  5.43 & 8.96  & 0.20 &  0.20 \\
     400 &  4.24 & 9.01  & 0.28 &  0.28 \\
     500 &  2.41 & 8.32  & 0.38 &  0.38 \\
     600 &  1.67 & 8.71  & 0.44 &  0.44 \\
     700 &  0.98 & 9.28  & 0.52 &  0.52 \\
     800 &  3.34 & 7.73  & 0.52 &  0.64 \\
     900 &  2.26 & 4.76  & 0.60 &  0.78 \\
     1000 & 4.10 & 6.01  & 0.61 &  0.83 \\
     \bottomrule
   \end{tabular}
 \end{table}




% Sämtliche im Folgenden durchgeführten Ausgleichsrechnungen werden mit der \emph{curve fit} Funktion aus dem für \emph{Python} geschriebenen package \emph{NumPy}\cite{scipy} durchgeführt. Fehlerrechnungen werden mit dem für \emph{Python} geschriebenen package \emph{Uncertainties}\cite{uncertainties} ausgeführt.

% % Examples
% \begin{equation}
%   U(t) = a \sin(b t + c) + d
% \end{equation}
%
% \begin{align}
%   a &= \input{build/a.tex} \\
%   b &= \input{build/b.tex} \\
%   c &= \input{build/c.tex} \\
%   d &= \input{build/d.tex} .
% \end{align}
% Die Messdaten und das Ergebnis des Fits sind in Abbildung~\ref{fig:plot} geplottet.
%
% %Tabelle mit Messdaten
% \begin{table}
%   \centering
%   \caption{Messdaten.}
%   \label{tab:data}
%   \sisetup{parse-numbers=false}
%   \begin{tabular}{
% % format 1.3 bedeutet eine Stelle vorm Komma, 3 danach
%     S[table-format=1.3]
%     S[table-format=-1.2]
%     @{${}\pm{}$}
%     S[table-format=1.2]
%     @{\hspace*{3em}\hspace*{\tabcolsep}}
%     S[table-format=1.3]
%     S[table-format=-1.2]
%     @{${}\pm{}$}
%     S[table-format=1.2]
%   }
%     \toprule
%     {$t \:/\: \si{\milli\second}$} & \multicolumn{2}{c}{$U \:/\: \si{\kilo\volt}$\hspace*{3em}} &
%     {$t \:/\: \si{\milli\second}$} & \multicolumn{2}{c}{$U \:/\: \si{\kilo\volt}$} \\
%     \midrule
%     \input{build/table.tex}
%     \bottomrule
%   \end{tabular}
% \end{table}
%
% % Standard Plot
% \begin{figure}
%   \centering
%   \includegraphics{build/plot.pdf}
%   \caption{Messdaten und Fitergebnis.}
%   \label{fig:plot}
% \end{figure}
%
% 2x2 Plot
% \begin{figure*}
%     \centering
%     \begin{subfigure}[b]{0.475\textwidth}
%         \centering
%         \includegraphics[width=\textwidth]{Abbildungen/Schaltung1.pdf}
%         \caption[]%
%         {{\small Schaltung 1.}}
%         \label{fig:Schaltung1}
%     \end{subfigure}
%     \hfill
%     \begin{subfigure}[b]{0.475\textwidth}
%         \centering
%         \includegraphics[width=\textwidth]{Abbildungen/Schaltung2.pdf}
%         \caption[]%
%         {{\small Schaltung 2.}}
%         \label{fig:Schaltung2}
%     \end{subfigure}
%     \vskip\baselineskip
%     \begin{subfigure}[b]{0.475\textwidth}
%         \centering
%         \includegraphics[width=\textwidth]{Abbildungen/Schaltung4.pdf}    % Zahlen vertauscht ... -.-
%         \caption[]%
%         {{\small Schaltung 3.}}
%         \label{fig:Schaltung3}
%     \end{subfigure}
%     \quad
%     \begin{subfigure}[b]{0.475\textwidth}
%         \centering
%         \includegraphics[width=\textwidth]{Abbildungen/Schaltung3.pdf}
%         \caption[]%
%         {{\small Schaltung 4.}}
%         \label{fig:Schaltung4}
%     \end{subfigure}
%     \caption[]
%     {Ersatzschaltbilder der verschiedenen Teilaufgaben.}
%     \label{fig:Schaltungen}
% \end{figure*}
