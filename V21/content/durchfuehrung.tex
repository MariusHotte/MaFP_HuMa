\section{Durchführung}
\label{sec:Durchführung}

Zu Beginn werden die optischen Elemente so positioniert, dass die Diodenspannung maximal wird. Dann wird der Versuchsaufbau parallel zu der horizontalen Komponente des Erdmagetfelds ausgerichtet und die vertikale Komponente durch die Helmholtsspule kompensiert. Durch die vorgenommenen Einstellungen minimiert sich die Breite der Resonanzkurve. 

In ersten Versuchsteil wird die angelegt Frequenz von $\SI{100}{\kilo\hertz}$ in $\SI{100}{\kilo\hertz}$-Schritten bis $\SI{1}{\mega\hertz}$ erhöht. Für jede Frequenz wird die Resonanzstelle der beiden Rubidium-Isotope gesucht. Hier ist anzumerken, dass durch die zwei vorhandenen Isotope auch zwei Resonanzstellen erzeugt werden. Für diesen Vorgang wird die Modulationsspule verwendet, falls diese jedoch bei höhreren Frequenzen nicht mehr ausreicht, wird zusätzlich ein horizontales Feld angelegt. Am Ende der Messreihe wird zusätzlich ein Bild des Kurvenverlaufs aufgenommen.

In zweiten Teil kommt ein zusätzlicher Funktionsgenerator zum Einsatz, der eine Rechteckspannung erzeugt mit einer Amplitude von $0-5\si{\volt}$ und einer Frequenz von $\SI{5}{\hertz}$. Die an- und absteigende Flanke des Kurvenlaufs der Tranparenz wird abfotograpiert. Danach wird die Frequenz der Rechteckspannung in $\SI{1}{\volt}$-Schritten im Bereich von $\SI{0,5}{\volt}-\SI{10}{\volt}$ varriert und für jede Frequenz wird die Zeit für einen Periodendurchlauf bestimmt.