\section{Durchführung}
\label{sec:Durchführung}

Im ersten Teil der Durchführung wird das Magnetfeld geeicht. Dazu wird mit einer Hallsonde die magnetische Flussdichte in Abhängigkeit der angelegten Stromstärke vermessen. \\
Der zweite Teil besteht aus der Justage des Versuchsaufbaus. Dafür wird mit einem Objektiv und einer Kondensorlinse das emittierte Licht auf einen Spalt fokussiert. Danach wird der Lichtstrahl auf einen Kollimator und anschließend auf Geradsichtprisma gelenkt. Hier ist zu beachten, dass der Durchmesser des Strahl nicht die breite des Primas überschreitet, da dies sonst ein Intensitätsverlust zur Folge hätte. Mit einer weiteren Linse wird erneut auf einen Spalt fokussiert. Ist dieser komplett geschlossen werden auf diesem die Spektralfarben Rot, Blau und Grün abgebildet. Durch eine Öffnung und Verschiebung des Spalts, kann somit einer dieser Spektrallinien ausgewählt werden. Zu diesem Zeitpunkt wird der Polarisator im Strahlengang platziert, wodurch zwischen den Übergängen $\Delta m=0$ und $\Delta m=\pm1$ unterschieden werden kann. Nach dem Lummer-Gehrcke-Prisma wird  zur Aufnahme der Aufspaltung der Spektrallinien eine Digitalkamera optimal justiert.
Nach dieser Justage, die mit der grünen Spektrallinie durchgeführt worden ist, wird anschließend die blaue und die rote Spektrallinie vermessen. Für jede einzelnenMessung wird dabei ein Bild ohne Magnetfeld und jeweils ein Bild mit der $\pi$- und $\sigma$-Aufspaltung aufgenommen.