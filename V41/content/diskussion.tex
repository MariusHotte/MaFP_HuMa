\section{Diskussion}
\label{sec:Diskussion}

Bei diesem Versuch gibt es einige Fehlerquellen. Zum einen die systematischen Fehler, die durch eine lineare Ausgleichsrechnung behoben werden und zum anderen systematische Fehler die bei der Durchführung auftreten können. Beispielsweise ist es unklar, ob die Entwicklung der Filme optimal durchgeführt wurde. Bei der Salzprobe war der Film zudem sehr stark belichtet, sodass die Reflexe nur schwer abgelesen werden konnten. Außerdem wurde ein Geodreieck zu Vermessung verwendet, wodurch eine Unsicherheit auf die abgelesenen Werte angenommen werden musste. 

Trotzdem konnte der Metallprobe 9 mit einer Abweichung von $\SI{1.038+-0.005}{\percent}
$ zum Literaturwert der Gitterkonstante $a$, dem Element Niob zugeordnet werden. Die im drauffolgenden untersuchte Salzprobe 5 konnte mit einer Abweichung von $\SI{-4.8+-0.4}{\percent}
$ Kaliumchlorid zugeordnet werden. 

% Bei der Analyse der Salzprobe konnte kein signifikanter Unterschied zwischen der Steinsalz/Fluorit- und der Cäsiumchlorid-Struktur festgestellt werden. Beim auftragen der Gitterkonstante $a$ gegen $\cos{(\theta)}$ zeigte sich nicht der erwartete lineare Verlauf. Somit ist anzunehmen, dass die Metallprobe 9 eine Steinsalz/Fluorit-Struktur besitzt. Mit dieser Annahme konnte das Salz 5 mit einer Abweichung von $\SI{-4.8+-0.4}{\percent}
$ als Kaliumchlorid identifiziert werden.