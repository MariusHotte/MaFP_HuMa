\section{Diskussion}
\label{sec:Diskussion}

Bei diesem Versuch gibt es einige Fehlerquelle. Zum einen die systematischen Fehler, die in Kapitel XXX beschrieben werden, die durch eine lineare Ausgleichsrechnung behoben werden und zum anderen systematische Fehler die bei der Durchführung auftreten können. Beispielweise ist es unklar, ob die Entwicklung der Filme optimal durchgeführt wurde. Bei der Salzprobe war der Film sehr stark belichtet, sodass die Reflexe nur schwer abgelesen werden konnten. Außerdem wurde ein Geodreieck zu Vermessung verwendet, wodurch eine Unsicherheit auf die abgelesenen Werte angenommen werden musste. 

Trotzdem konnte der Metallprobe 9 mit einer Abweichung von $\SI{1.038+-0.005}{\percent}
$ zum Literaturwert der Gitterkonstante $a$, dem Element Niob zugeordnet werden. Obwohl für die Zuordnung der Gitterstruktur bei der Salzprobe ein anderes Verfahren benutzt wurde, konnte auch hier mit einer Abweichung von $\SI{2.46}{\percent}$ zum Literaturwert, das Salz CsJ (Caesiumiodid) identifiziert werden.