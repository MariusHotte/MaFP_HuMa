\section{Durchführung}
\label{sec:Durchführung}
Der zylindrische Stab wird zuerst mit einer gleichmäßigen Schicht Vaseline bestrichen.
 Die verwendeten Proben sind unbekannte Metalle oder Salze. Die Probe wird in pulverisierter
  Form auf in einer möglichst ebenen Schicht auf den Stab aufgetragen. Es ist eine möglichst
   große Menge der Probe gleichmäßig auf dem Stab zu verteilen. Im nächsten Schritt
    wird der Stab in die Versuchsapparatur eingespannt, sodass die Röntgenstrahlen
     mittig auf den Stab treffen. Hierbei ist ein besonderes Augenmerk auf die Justierung zu legen.
      Anschließend wird unter Rotlicht zwei Löcher in den Filmstreifen gestanzt und in die Apparatur eingespannt.
       Danach wird das Metall 2 Stunden und das Salz 4 Stunden mit monochromatischen Röntgenstrahlen bestrahlt.
       Zum Schluss wird der Filmstreifen entsprechend der Anleitung im Fotolabor entwickelt.
