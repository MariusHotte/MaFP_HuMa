\section{Diskussion}
\label{sec:Diskussion}
Die transversal und longitudinal Komopnente der Relaxationszeit konnten
mit kleinen Fehler bestimmt werden:
$$(T_1=2.1 \pm 0.2 \,)\, \text{s} \quad \quad T_2=1.4 \pm 0.2 \,)\, \text{s}$$
Die Messwerte weisen eine gute Übereinstimmung, bis ein
einzelne Ausreißer, mit den Fit auf.\\
Der bestimmte Feldgradient $G=-0.41$ liegt unter dem erwarteten Wert,
von $\approx G=-1\,\text{T}\text{m}^{-1}$. Sodass die damit bestimmte
Diffusionskonstante große Abweichungen $\approx 53\, \%$ vom Literaturwert $D_\text{lit}=2.1\cdot
10^{-9}\,\text{m}^2\text{s}^{-1}$\cite [wikidiff].\\
Der Verlgeich der bestimmten Viskosität $(\eta=0.949)\,$mPa$\,$s
mit dem Theoriewert $(\eta=0.890)\,$mPa$\,$s\cite[visko] zeigt eine Abweichung von $\approx
18\,\%$ eine leichte Abweichung.\\
Der berechnete Radius des Wassermolekühls $(16.8 \pm 0.6)\,$m
lässt sich mit einer Abschätzung zum Molekühlradius,
welche nur von der Dichte, dem Molekulargewicht $m=18.01528\,$g/mol  \cite[visko]
und der Annahme der dichten kexagonalen Kugelpackung abhängt. Für
die dichte Kugelpackung wird $n=74\%$ verwendet, es ergibt sich für
die Abschätzung des Molekülradius:
$$R=\left( \frac{3m}{4\pi \rho \cdot 0.74}\right)^{1/3}\approx 2.13\cdot 10^{-11}\,\text{m}$$
Der berechnete Wert und die Abschätzung des Radius $R$ weißen eine
Größenordnung Unterschied auf. Diese könnte auf sich durch die
ungenauen Werte für die Diffusionskonstante und auf den Feldgradienten
zurückführen lassen. Des Weiteren wird hier nur eine Abschätzung des Radius
$R$ vorgenommen. Dennoch sollte das Ergebnis in der selben
Größenordnung liegen. 
