\section{Diskussion}
\label{sec:Diskussion}
Zu Beginn wurde eine amplituden Modulation mit Hilfe eines Schwingkreises
durchgeführt. Diese konnte erfolgreich durchgeführt werden, sodass
eine Analyse der Frequenzspektrums möglich war. Am analysierten
Frequenzspektrum konnten die Peaks der Frequenzen $\nu_\text{T}\pm \nu_\text{m}$
identifiziert werden. Außerdem ließ sich am Frequenzspektrum die Trägerunterdrückung
erkennen, da diese Frequenz mit einer kleinen Amplitude auftrat.\\
Anschließend  wurde das amplitudenmodulierte Signal mit Trägerabstrahlung
erzeugt. Im Gegensatz zur Trägerunterdrückung führt der Effekt der Trägerabstrahlung
dazu, dass sich höhere Schwingungsmoden besser erkennen lassen und
vermessen werden können. Dieser Effekt ist auf die Nichtlinerarität der
Diode zurückzuführen. Durch die Werte der maximalen und minimalen Spannung
ließ sich der Modulationsgrad auf $m=0.2934$ bestimmen.\\
Alternativ wurde der Modulationsgrad über das Frequenzspektrum bestimmmt.
Hierfür ergab sich ein Wert von $m=0.45$. Die berechneten
Werte der beiden Methoden stimmen nicht überein. Ein möglicher
Grund für diese Abweichung sind die im Zeitsignal auftretenden Oberwellen
erklären. Diese werden sowohl im Frequenzspektrum als auch in der Theorie
nicht berücksichtigt. Dieses könnte erkläen, warum der Modulationsgrad aus
der Frequenzanalyse größer ist.\\
Anschließend wurde eine Modulationsgradbestimmung für die frequenzmodulierten
Schwingungen durchgeführt. Hier konnte der Modulationsgrad auf
$m=0.606 \pm 0.002$ festgelegt werden.  Der daraus resultierende
Frequenzhub liegt bei $\varDelta \nu=(76.4 \pm 0.2)\,$kHz.\\
Ebenfalls wurde hier eine Frequenzanalyse zur Bestimmung des Modulationsgrades
durchgeführt. Der damit bestimmte Modulationsgrad liegt bei $m=0.0574 \pm 0.003$
Die Abweichungen der verschiedenen Modulationsgrade liegt bei $5\%$.
Die Überprüfung der Phasenabhängigkeit $\cos(\phi)$  des Ringmodulators wurde
mit Hilfe einer Funktionsanpassung durchgeführt. Es fiehl auf, dass die Messwerte
alternierend linear ansteigen und abfallen. Dennoch ließ sich der $\cos(\phi)$
gut durch die Messwerte legen und die Fitparameter liegen im Rahmen der These.
Der Offset $U_\text{off}\approx 0$ und $b\approx 1$, die Amplitude $U_0$ spielt
keine Rolle zur Überprfung der Abhängigkeit.
Die Demodulationen wurden mit verschiedenen Methoden durchgeführt. Bei der
Demodulation mit Hilfe eines Ringmodulators ist ein moduliertes Signal
erfolgreich demoduliert worden, sodass das Ausgangssignal kenntlich gemacht wurde. \\
Anschließend wurde eine Demodulation des zuvor modulierten Signals mit Hilfe
einer Gleichrichterdiode demoduliert. Bei dieser Demodulation ist ein
leichter Phasenshift zu erkennen. Des Weiteren wird druch die Gleichrichterdiode
die Frequenz des demodulierten Signals verdoppelt, was zu erwarten war und durch
die Messungen bestätigt wurde. \\
Nach den amplitudenmodulierten Signal wird zuletzt das frequenzmodulierte Signal untersucht.
Hier konnte auch analog zu den ersten Demodulationen das Modulationssignal
erfolgreich extrahiert werden. Was hier besonders deutlich wurde ist der
Amplitudenverlust durch die Demodulation.  Des Weitern wurde sich durch den
Tiefpass ein leichter Bias eingefangen.
