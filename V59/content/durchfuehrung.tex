\section{Durchführung}
\label{sec:Durchführung}

\subsection{Amplitudenmodulierte Schwingungen}
Zu Beginn soll eine amplitudenmodulierte Schwingung erzeugt werden. Dafür werden die Trägerfrequenz $\omega_T$ und die Modulationsfrequenz $\omega_M$ so gewählt, dass die Trägerfrequenz etwa um einen Faktor 100 hochfrequenter ist als die Modulationsspannung. Für die anschließende Erzeugung wird ein Ringmodulator nach Kapitel \ref{sec:Ringmodulator} verwendet. Um die Korrektheit der amplitudenmodulierten Schwingungen zu überprüfen, wird an einem weiteren Kanal des Oszilloskops zusätzlich die Modulationsfrequenz angelegt. Im folgenden wird das Frequenzspektrum mittels Frequenzanalysator untersucht und das Ergebnis von dem Bildschirm des Frequenzanalysators abphotographiert.

Anschließend wird die Erzeugung einer amplitudenmodulierten Schwingung mit der in Kapitel \ref{sec:Primitive_Modulatorschaltung_mittels_Diode} vorgestellten Diodenschaltung untersucht. Dabei können die eingestellten Frequenzen beibehalten werden. Es wird ebenfalls mittels Frequenzanalysator das Spektrum untersucht und dokumentiert.

\subsection{Frequenzmodulierte Schwingungen}
Für die Erzeugung wird die in Kapitel \ref{sec:Frequenzmodulator_mit_geringem_Frequenzhub} beschriebene Schaltung verwendet. Dabei muss auf die korrekte Phasenverschiebung, verursacht durch das Laufzeitkabels, geachtet werden. Mit einer gegebenen Periodendauer von $\SI{250}{\ns}$ ergibt sich die benötigte Frequenz zu $\omega=\SI{1}{\mega\hertz}$. An dem Oszilloskop werden Einstellungen vorgenommen, sodass die Verschmierung der modulierten Spannung über eine Periode zu erkennen ist. Mit Hilfe der "Curser"-Funktion wird zusätzlich die maximale Breite in x-Richtung der Verschmierung in mitten einer Periode bestimmt. Alle Zwischenschritte werden dabei dokumentiert und das aufgenommene Frequenzspektrum am Frequenzanalysator photographiert.

\subsection{Gleichrichtereigenschaften des Ringmodulators}
Mit der in Kapitel \ref{sec:Phasenempfindlichkeit_der_Gleichrichterdiode} beschriebenen Schaltung wird die Ausgangsspannung des Ringmodulators in Abhängigkeit der Frequenz des Eingangssignals mit Hilfe eines Voltmeters vermessen. Die sich ergebende Proportionalität zum Cosinus der Phase $\phi$ wird solange vermessen, bis die aufgenommenen Daten für eine volle Periode ausreichen.

\subsection{Demodulation einer amplitudenmodulierten Schwingung}
In diesem Abschnitt soll eine amplitudenmodulierte Schwingung mittels der in Kapitel \ref{sec:Demodulation_amplitudenmodulieter_Schwingungen} beschriebenen Schaltung demoduliert werden. Erneut wird mit dem zweiten Kanal des Oszilloskops überprüft, ob die demodulierte Schwingung der ursprünglichen Modulationsfrequenz übereinstimmt. Auch hier wird wieder anschließend das Frequenzspektrum mit Hilfe des Frequenzanalysator aufgezeichnet.
\\
Ebenfalls soll die amplitudenmodulierte Schwingung, die durch die Diodenschaltung nach Kapitel \ref{sec:Primitive_Modulatorschaltung_mittels_Diode} erzeugt wurde, demoduliert werden. Hierzu wird die Schaltung aus Kapitel \ref{DSmeGD} verwendet. Diese Mal werden auch die Zwischenschritte(nach der Diode und nach dem Tiefpass) dokumentiert, sodass die einzelnen Vorgänge besser nachvollzogen werden können.

\subsection{Demodulation einer frequenzmodulierten Schwingung}
Als letztes soll die frequenzmodulierten Schwingung, durch den in Kapitel \ref{sec:Demodulation_frequenzmodulieter_Schwingungen} beschrieben Aufbau demoduliert werden. Hierfür muss die Kapazität des Kondensator so gewählt werden, dass sich die Frequenzen der Seitenbänder auf der Flanke der Resonanzkurve befinden. Somit müsste eine amplitudenmodulierte Schwingung am Oszilloskop messbar sein. Auch hier werden wieder die Teilschritte (nach der Diode und nach dem Tiefpass) genau dokumentiert. Schlussendlich wird erzeugte demodulierte Spannung mit der ursprünglichen Modulationsspannung vergleichen. 