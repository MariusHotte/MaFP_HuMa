\section{Theorie}
\label{sec:Theorie}

Damit eine elektromagnetische Welle zur Informationsübertragung verwendet werden kann, wird eine hochfrequenten Trägerspannung mit einer niederfrequenten Modulationsspannung kombiniert. Diese Kombination kann zum einen durch die Modulierung der Frequenz oder zum anderen durch die Modulierung der Amplitude beider Spannung erzeugt werden. Unter Verwendung mehrerer Trägerspannungen ist es möglich verschiedene Informationen gleichzeitig zu übertragen, wie z.B. bei den Funkkanälen des Radios. Zur Extraktion der Information der Welle am Empfangsort wird die Demodulation angewandt. Die Grundlagen der zuvor angesprochenen Frequenz- und Amplitudenmodulation werden in Kapitel \ref{sec:Amplitudenmodulation} und \ref{sec:Frequenzmodulation} erläutert.

\subsection{Amplitudenmodulation}
\label{sec:Amplitudenmodulation}

Bei dieser Methode der Modulation führt die Amplitude einer hochfrequenten Trägerspannung Schwingungen im Rhythmus der niederfrequenten Modulationsspannung, die die Informationen enthält, aus. Beschrieben werden die beiden verwendeten Spannungen durch die Gleichungen 
\begin{align}
	U_T(t)=\hat{U}_T\cos{(\omega_Tt)} \;\; \text{und} \;\; U_M(t)=\hat{U}_M\cos{(\omega_Mt)}\; ,
\end{align}
dabei bezeichnet $U_T$ die Trägerspannung, $U_M$ die Modulationsspannung, $\omega_T$ die Trägerfrequenz und $\omega_M$ die Modulationsfrequenz. Durch die Kombination der Spannungen ergibt sich
\begin{align}
	U(t)=\hat{U}_T\Big(1+m\cos{(\omega_M)}\Big)\cos{(\omega_Tt)}\;.
	\label{eq:1}
\end{align}

Der mit $m$ bezeichnete Modulationsgrad ist über die Amplitude der Modulationsspannung wie folgt definiert:
\begin{align}
	m=\gamma\hat{U}_M \;\; \text{mit} \;\; m= [0,1]
\end{align} 

In Abbildung \ref{fig_T1} ist die amplitudenmodulierte Spannung des Signals, mit den Extremwerten 
\begin{align}
	\hat{U}_T(1-m) \;\; \text{und} \;\; \hat{U}_T(1+m)\;,
\end{align}
dargestellt. 

\begin{figure}
    \centering
    \includegraphics[width=0.55\textwidth]{ressources/T1.png}
    \caption{Darstellung einer amplitudenmodulierten Spannung in Abhängigkeit der Zeit $t$\cite{skript}.}
    \label{fig_T1}
\end{figure}

Um das Frequenzspektrum dieser Schwingung zu ermitteln, wird Gleichung \eqref{eq:1} mittels trigonometrischen Beziehungen zu
\begin{align}
	U(t)=\hat{U}_T\left(\cos{(\omega_Tt)+\frac{1}{2}m\cos{\big((\omega_T+\omega_M)t\big)}}+\frac{1}{2}m\cos{\big((\omega_T-\omega_M)t\big)}\right)
\end{align}
vereinfacht. Somit besteht das Frequenzspektrum aus den drei Frequenzen
\begin{align}
	\omega_T-\omega_M,\quad\omega,\quad\omega_T+\omega_M\;.
\end{align}
In Abbildung \ref{fig_T2} sind die zuvor beschriebenen Kreisfrequenzen gegen die Spannung aufgetragen.

\begin{figure}
    \centering
    \includegraphics[width=0.55\textwidth]{ressources/T2.png}
    \caption{Darstellung eines Frequenzspektrums einer amplitudenmodulierten Spannung\cite{skript}.}
    \label{fig_T2}
\end{figure}

Bei der Betrachtung des Spektrums ist zu beachten, dass, wie auch in Abbildung \ref{fig_T2} zu sehen, die äußeren beiden Frequenzen, die sogenannten Seitenbänder, identische Informationen enthalten. Im Gegensatz dazu enthält die Frequenz $\omega_T$, die sogenannte Trägerabstrahlung, durch die fehlende Abhängigkeit von $\omega_M$ keine Information. Besteht der Umstand, dass $\omega_M$ nicht nur aus einer sondern aus einer Kombination mehrerer Frequenzen besteht, verbreitern sich die Seitenbänder in Abbildung \ref{fig_T2}. Nachteile der Amplitudenmodulation sind unter anderem die geringe Störsicherheit und die geringe Verzerrungsfreiheit. Beispielsweise kann es in engen Schaltkreise mit mangelnder Abschirmung zur Induktion einer Fremdspannung kommen, wodurch die Amplitude der Modulation beeinflusst wird. Unbeeinflusst bleibt die Frequenz, weshalb im Folgenden die Frequenzmodulation beschrieben wird.

\subsection{Frequenzmodulation}
\label{sec:Frequenzmodulation}
Bei der Frequenzmodulation entsteht eine Änderung der Trägerfrequenz verursacht durch ein Modulationssignal. Dabei ist zu betonen, dass nur die Frequenz eine Variation erfährt und die Amplitude konstant bleibt. Die modulierte Spannung kann wie folgt dargestellt werden:
\begin{align}
	U(t)=\hat{U}\sin{\left(\omega_Tt+m\frac{\omega_T}{\omega_M}\cos{(\omega_Mt)}\right)}
	\label{eq:2}
\end{align}
Die Momentanfrequenz wird dabei beschrieben durch
\begin{align}
	f(t)=\frac{\omega_T}{2\pi}\Big(1-m\sin{(\omega_Mt)}\Big)\;.
\end{align}
Mit dem Frequenzhub
\begin{align}
	\Delta f=\frac{m\omega_T}{2\pi}
\end{align}
wird die Variationsbreite, also ein Maß für die Variation der Frequenz definiert. In Abbildung \ref{fig_T3} ist beispielhaft eine frequenzmodulierte Schwingung skizziert.

\begin{figure}
    \centering
    \includegraphics[width=0.85\textwidth]{ressources/T3.png}
    \caption{Darstellung einer frequenzmodulierten Spannung in Abhängigkeit der Zeit $t$\cite{skript}.}
    \label{fig_T3}
\end{figure}

Anhand einer sogenannten Schmalband-Frequenzmodulation, bei der gilt 
\begin{align}
	m\frac{\omega_T}{\omega_M}<<1,
\end{align}
soll im folgende grundlegende Eigenschaften der Frequenzmodulation verdeutlicht werden. Mittels trigonometrischer Funktionen wird Gleichung \eqref{eq:2} umgeformt zu
\begin{align}
	U(t)=\hat{U}\left(\sin{(\omega_Tt)}\cos{\left(m\frac{\omega_T}{\omega_M}\cos{(\omega_Mt)}\right)} + \cos{(\omega_Tt)}\sin{\left(m\frac{\omega_T}{\omega_M}\cos{(\omega_Mt)}\right)}\right) \:.
\end{align}
Mit einer Reihenentwicklung der Sinus- und Cosinustherme, bei der nur Therme erster Ordnung betrachtet werden, entsteht die folgende Gleichung
\begin{align}
	U(t)=\hat{U}\left(\sin{(\omega_Tt)+\frac{1}{2}m\frac{\omega_T}{\omega_M}\cos{\big((\omega_T+\omega_M)t\big)}}+\frac{1}{2}m\frac{\omega_T}{\omega_M}\cos{\big((\omega_T-\omega_M)t\big)}\right)\;,
\end{align}
die sehr große Ähnlichkeit zur amplitudenmodulierten Spannung aufweist, da auch hier das Frequenzspektrum aus den drei Frequenzen
\begin{align}
	\omega_T-\omega_M,\quad\omega,\quad\omega_T+\omega_M\;
\end{align}
besteht. Der wesentliche Unterschied besteht jedoch in der Trägerschwingung, daraus resultiert eine Phasenverschiebung um $90^{\circ}$ der Sinusfunktion.
Unter Berücksichtigung höherer Therme in der Reihenentwicklung ergibt sich 
, daraus resultiert eine Phasenverschiebung um $90^{\circ}$ der Sinusfunktion
\begin{align}
	U(t)=\hat{U}\sum_{n=-\infty}^{\infty} J_n\left(m\frac{\omega_T}{\omega_M}\right)\sin{\big((\omega_T+n\omega_M)t\big)}\;. 
	\label{eq:3}
\end{align}
Der Ausdruck $J_n(x)$ steht dabei für die Besselfunktion n-ter Ordnung. Bei der Betrachtung von Gleichung \eqref{eq:3} fällt auf, dass der Iterator der Summe in der Sinusfunktion ein beliebig großes Frequenzspektrum ermöglicht. Da jedoch die Besselfunktion für steigende $n$ gegen Null strebt, müssen nur Frequenzen nahe der Trägerfrequenz $\omega_T$ berücksichtigt werden.


% 2x2 Plot
% \begin{figure*}
%     \centering
%     \begin{subfigure}[b]{0.475\textwidth}
%         \centering
%         \includegraphics[width=\textwidth]{Abbildungen/Schaltung1.pdf}
%         \caption[]%
%         {{\small Schaltung 1.}}
%         \label{fig:Schaltung1}
%     \end{subfigure}
%     \hfill
%     \begin{subfigure}[b]{0.475\textwidth}
%         \centering
%         \includegraphics[width=\textwidth]{Abbildungen/Schaltung2.pdf}
%         \caption[]%
%         {{\small Schaltung 2.}}
%         \label{fig:Schaltung2}
%     \end{subfigure}
%     \vskip\baselineskip
%     \begin{subfigure}[b]{0.475\textwidth}
%         \centering
%         \includegraphics[width=\textwidth]{Abbildungen/Schaltung4.pdf}    % Zahlen vertauscht ... -.-
%         \caption[]%
%         {{\small Schaltung 3.}}
%         \label{fig:Schaltung3}
%     \end{subfigure}
%     \quad
%     \begin{subfigure}[b]{0.475\textwidth}
%         \centering
%         \includegraphics[width=\textwidth]{Abbildungen/Schaltung3.pdf}
%         \caption[]%
%         {{\small Schaltung 4.}}
%         \label{fig:Schaltung4}
%     \end{subfigure}
%     \caption[]
%     {Ersatzschaltbilder der verschiedenen Teilaufgaben.}
%     \label{fig:Schaltungen}
% \end{figure*}
