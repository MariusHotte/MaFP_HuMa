\section{Grundlagen}
\label{sec:Grundlagen}
Seit der ersten theoretischen Beschreibung der stimulierten Emission von Albert Einstein 1916 und dem experimentelle Beweis 1928 von Rudolf Ladenburg wurde er stets weiterentwickelt. Die Anwendungsmöglichkeiten beschränken sich nicht nur auf die verschiedensten Zweige der Wissenschaft sondern findet ebenfalls Verwendung im Alltag. Der letzte große Erfolg wurde 2018 mit einem Nobelpreis ausgezeichnet, da es gelungen war mit Hilfe eines Lasers eine optische Pinzette zu realisieren.
In den folgenden Kapitel wird zunächst auf die Grundlagen und die essenziellen Bestandteile einen Lasers eingegangen, gefolgt von der Beschreibung des Diodenlasers, welcher in diesem Experiment verwendeten wurde.

\subsection{Laser}
Wird ein System angeregt kehrt es nach einer gewissen Zeit in seinen Grundzustand zurück. Der Energieunterschied zwischen den beiden Zuständen wird bei der Rückkehr in den Grundzustand in Form eines Photons oder eines Phonons ungewandelt, welches emittiert wird. Die Emission eines Photons kann dabei spontan oder stimuliert entstehen. Der spontane Übergang in ein niedrigeres Niveau ist ein zufälliger, ungetriebener Prozess. Die stimulierte Emission wird Gegensatz dazu, durch ein externes Photon ausgelöst. Weißt dieses Photon genau die Energie der Energiedifferenz zwischen zwei Niveaus aus, geht das System in das niedrigere Niveau über und emittiert dabei ein Photon gleicher Wellenlänge und Phase. Den Zustand, das sowohl die Wellenlänge als auch die Phase überstimmen, wird als Kohärenz bezeichnet und ist die Grundlage jedes Lasers.

\subsubsection{Bestandteile}
Jeder Laser benötigte eine Pumpen, ein aktives Medium, einen Resonator und eine Kühlung.

Die Pumpe ist dafür die Elektronen des aktiven Mediums, die sich im Grundzustand befinden, anzuregen, sodass die in einen Zustand höherer Energie übergehen. In Abbildung XXX ist dieser Vorgang an einem Drei-Niveau-System dargestellt. Die Elektronenverteilung wird dabei über die Fermifunktion
\begin{align}
	f(E)=\frac{1}{e^{\frac{E-\mu}{k_B T}}+1}
\end{align}
beschrieben. Diese ist Abhängig von der Energie der Elektronen $E$, dem chemischen Potential $\mu$, der Bolzmannkonstante und der Temperatur des Systems $T$. Bei fallender Temperatur gehen somit immer mehr Elektronen in der Grundzustand über, bis der Grenzfall erreicht ist und ausschließlich ein Niveau besetzt ist. Im gegenteiligen Grenzfall konvergiert die Funktion gegen $1/2$, wodurch das angeregte Zustand als auch der Grundzustand gleich besetzt sind. 

Für die zuvor angesprochene stimulierte Emission muss eine Besetzungsinversion realisiert werden. Dieses ist mit einem System mit drei oder mehr Energieniveaus möglich. Dabei, wie in Abbildung XXX angedeutet, erfolgt die Relaxation in das zweite Energieniveau wesentlich schneller als die Relaxation vom zweiten Niveau in das erste. Somit befindet sich ein Großteil der Elektronen im zweiten Niveau. Der Abstand der einzelnen Energieniveaus ist über das verwendete aktive Medium definiert. Aus diesem Abstand ergibt sich die Wellenlänge des emittierten Photons. Die Spiegel, die das aktive Medium einschließen, reflektieren die entstanden Photonen und erzeugen stehende Wellen. Hierfür müssen die folgenden Bedingungen erfüllt sein:
\begin{align}
	\lambda=\frac{2Ln}{m}\qquad \text{und} \qquad\nu=\frac{cm}{2Ln}\:, \quad \text{mit}  \quad m \in \mathbb{N} 
\end{align}
Zwischen den verstärkten Frequenzen besteht ein äquidistanter Abstand von
\begin{align}
	\Delta \nu=\frac{c}{2Ln}.
\end{align}
Für den Grenzfall $\nu >>\Delta\nu$ besteht ein proportionaler Zusammenhang zwischen der Wellendifferenz
\begin{align}
	\Delta \lambda =\lambda_2- \lambda_1=c\left(\frac{1}{\nu_2}-\frac{1}{\nu_1}\right) \approx \frac{c}{\nu^2}\Delta\nu.
\end{align}
und der Frequenzunterschied. Einer der verwendeten Spiegel reflektiert nur etwa $\SI{95}{\percent}$, wodurch der entstandene Laserstrahl das aktive Medium verlassen kann. Die verwendete Kühlung sorgt für eine konstante Temperatur des aktiven Mediums und somit für eine stabile Emission.

\subsection{Diode}

Eine Diode ist ein sogenannter Halbleiter. Das bedeutet, dass durch thermische Anregungen Valenzelektronen vom Valenzband in das Leitungsband übergehen können. Die Ursache dafür liegt in der Größe der Bandlücke, die sich im Bereich von $\SI{0.4}{\electronvolt}$ und  $\SI{4}{\electronvolt}$. Außerhalb dieses Bereichs wird das Material als metallischer Leiter oder als Isolator bezeichnet. Im Falle eines elektrischen Leiters überlappen sich die zuvor angesprochenen Bänder, wodurch keine externe Anregung für eine Übergang nötig ist. Im Gegensatz dazu ist die Bandlücke eines Isolators so groß, dass keine externe Anregung ausreicht, um ein Valenzelektron in das Leitungsband zu befördern. 

Für einen p-n-Übergang ist es nötigt ein Halbleiter-Material, meist Silizium, zusätzlich zu dotieren. Bei diesem werden dem Halbleiter Fremdatome eine höheren oder einer niedrigeren Hauptgruppe hinzugefügt. Somit besitzt n-dotiertes Material einen Überschuss an Elektronen und p-dotiertes Material einen mangel an ELektronen, also effektiv einen Überschuss an Löchern, die als Quasiteilchen behandelt werden. 

Bei Kontakt entsteht der sogenannte p-n-Übergang. Dabei driften Elektronen in das p-dotierte und Löcher in das n-dotierte Halbleitermaterial. Durch die Rekombinationen in der Grenzschicht verbleiben ionisierte Atome, die nach und nach ein elektrisches Feld aufbauen und somit dem Drift weiterer freier Ladungsträger entgegenwirkt, bis schlussendlich ein Gleichgewicht entsteht. Die Ausdehnung der Depletionszone kann durch das Anlegen einer externen Spannung sowohl verkleinert also auch vergrößert werden. Bei Anlegen einer positiven Spannung ($V>0$) an die n-dotierte Seite des Halbleiters werden weitere freie Ladungsträger aus dem Material gezogen, wodurch sich die Depletionszone ausdehnt. Im Gegensatz dazu verkleinert sich die Depletionszone, wenn eine negative Spannung ($V<0$) angelegt wird. Die Depletionstiefe
\begin{align}
	d=\sqrt{\frac{2\epsilon\cdot(U_D-U)(N_D+N_A)}{q\cdot N_D\cdot N_A}}
\end{align} 
ist dabei sowohl von der Akzeptor- und Donatorkonzentration $N_{A/D}$, welche die Anzahl an eingebrachten Atomen zur n- und p-Dotierung pro Volumen beschreibt, als auch von der Dielektrizitätskonstante $\epsilon$ und der Elementarladung $q$ abhängig. Der Strahlungsrekombinationsstrom der für die Emission von Photonen verantwortlich ist, steigt proportional zur Spannung, weshalb die Diode für diesen Versuch in Vorwärts-Richtung (V>0) betrieben wird.

\subsection{Dioden Laser}
Die in Kapitel XXX angesprochenen essentiellen Komponenten eines Lasers, sind auch bei einem Diodenlaser zu finden. Der Halbleiter mit deinem p-n-Übergang wird als aktives Medium verwendet. Die entstehenden Bandlücke definiert die Wellenlänge der emittierten Photonen. Der angelegte Strom sorgt für den Pump-Prozess, da dieser Elektronen und Löcher einbringt und damit eine Art Besetzungsinversion zwischen dem Leitungsband und dem Valenzband hervorruft. Wie in Abbildung XXX dargestellt, dient der Halbleiter selbst als Resonator und bildet eine optischen Wellenleiter aufgrund seinem Reflektionskoeffizienten. 

%BILD

Der entstandene Laser verlässt auf zwei entgegengesetzten Oberflächen die Diode mit einem elliptisch geformten Strahl, der sehr stark divergiert. Um die Sensitivität für andere Wellenlängen zu mindern, wird zusätzlich noch ein externer Resonator verwendet, auf den in späteren Kapiteln genauer eingegangen wird.

Für niedrige Ströme sind die Verluste im Medium so hoch, dass keine kohärente Strahlung stattfinden kann. Damit dominiert in diesem Bereich die spontane Emission mit einem breiten Spektrum, vergleichbar mit der Emission einer handelsüblichen LED. Wird der Grenzstrom $I$, der sogenannte Threshold, überschritten domoniert die kohärente Strahlung, wodurch der gewünscht Laserstrahl zustande kommt.

\section{TeachSpin Diodenlaser (Sanyo DL-2140-201S)} 

In den folgenden Kapiteln werden die Eigenschaften des verwendeten Lasers und der Aufbau des Experiments erläutert.

\subsection{Die Wellenlänge}
Die endgültige Wellenlänge des Lasers hängt stark von den verwendeten Komponenten und deren Konfigurationen ab. In der Abbildung XXX ist die Laserausbeute, der sogenannte Net Gain gegen die Wellenlänge aufgetragen.

%BILD

Das Spektrum des aktiven Mediums ist sehr bereit und weißt eine hohe Ausbeute über einige Wellenlänge auf, ist jedoch stark Temperatur und Stromabhängig. Wie in Abbildung XXX dargestellt kann ein Wellenlängenshift durch einen Temperaturanstieg verursacht werden. Dieses wird durch eine Verkleinerung der Depletionszone bei hohen Temperaturen verursacht. Die Wellenlänge $\lambda_0$ kann ebenfalls verschoben werden durch eine Änderung des Diodenstroms. Die eingezeichneten Linien haben dabei eine Steigung von $\SI{200}{\mega\hertz\per\milli\volt}$ und spiegeln die umsetzbaren Wellenlängen wieder, da sich im Resonator nur eine diskrete Anzahl von stehende Wellen ausbilden kann. Somit existieren zwei Varianten die Wellenlänge der ursprünglichen Mode des Lasers zu verändern. In der späteren Durchführung wird hierfür der Strom der Diode verwendet, da dieses eine instantane Verschiebung verursacht im Gegensatz einer Temperaturänderung.

Der Resonator hat eine Länge von $L=\SI{700}{\micro\meter}$ und einen Reflexionskoeffzienten von $n=3.6$. Der Abstand zwischen dem vom Resonator unterstützen Frequenzen wird auch als $free spectral range$ bezeichnet und ist für den verwendeten Laser mit $\Delta\nu_{FSR}=\SI{60}{\giga\hertz}$ angebenen. Mit Hilfe von Gleichung von XXX ergibt sich somit der Abstand zwischen den Wellenlängen $\Delta\lambda_{FSR}=\SI{0.122}{\nano\meter}$.

Das zuvor in Kapitel XXX erwähnte Gitter wird wie in Abbildung XXX dargestellt angeordnet. Dabei wird etwas $\SI{15}{\percent}$ des Lasers zurück in den Resonator reflektiert, welches zu einer Stabilisierung des Strahls führt. Analog wie zu einem Prisma wird nur unter einem bestimmten Winkel eine bestimmte Wellenlänge reflektiert, weshalb in Abbildung XXX nur ein einzelnes Maximum dargestellt. Der Winkel des Gitters wird so gewählt, dass das Maximum erster Ordnung des Interferenzmusters reflektiert wird. Die Breite dieses Maximums
\begin{align}
	\Delta \nu= \nu/N
\end{align}
ist abhängig von der reflektierten Frequenz und der Anzahl der Gitterlinie die vom Laserspot überdeckt werden. Mit den spezifischen Parametern ergibt sich hier eine Frequenzbreite von $\Delta \nu = \SI{70}{\giga\hertz}$

Die Länge des externe Resonators kann durch einen Piezokristall modifiziert werden. Da der externe Resonator (L=$\SI{15}{\milli\meter}$) wesentlich größer als der interne Resonator (L=$\SI{700}{\micro\meter}$) ist, sind im Vergleich mehr Frequenzen zu realisieren. Aus diesem Grund ist das Signal in Abbildung XXX hochfrequenter dargestellt. 

\subsection{Interne und externe Moden} 
In Abbildung XXX ist erneut die Laserausbeute (Gain) gegen die Wellenlänge aufgetragen, jedoch unter verschiedenen Modenkonstellationen. Die Maxima oder auch Moden des internen Resonators werden als Int0 oder Int1 und die Moden des externen Resonators e0, e1 oder e2 bezeichnet. In Abbildung a) stimmt die Wellenlängen des internen und des externen Resonators überein, wodurch die Frequenz favorisiert wird, da sich der Gain addiert. Durch eine Längenänderung des externen Resonators verursacht durch die Ausdehnung des Piezokristalls erfolgt eine Verschiebung der Wellenlänge, wie in Abbildung b) zu sehen. Bei dieser Ausrichtung wird keine einzelne Frequenz favorisiert, da sich die Moden e0 und e1 genau zwischen Int0 befinden. Erfolgt eine weitere Verschiebung der Frequenz wird die nächste Mode (siehe c) ) verstärkt. Dieser Vorgang des "Modenwechselns" wird auch als Mode-hop bezeichnet. Ebenfalls einen Modenwechsel von Int0 auf Int1, aufgrund der Änderung der Stromstärke, wird als Mode-hop bezeichnet.

Werden sowohl die Stromstärke als auch das Volumen des Piezokristalls simultan verändert, findet für beide Moden eine Wellenlängenverschiebung statt. Mit diesem Vorgehen ist es möglich ein kontinuierliches Frequenzspektrum mit dem Laser zu durchfahren, da kein Mode-Hop verursacht wird. Das kontinuierliche Spektrum ermöglicht beispielsweise verschiedene Anregungszustände eines Atoms zu untersuchen, welches im folgenden anhand von Rubidium genauer diskutiert wird.

\subsection{Rubidium}

Mit dem nun geschaffenen kontinuierlichen Frequenzspektrum ist es möglich Absorptionslinien von Rubidium mit Hilfe der Fluoreszenz zu beobachten. 

Fluoreszenz tritt auf wenn ein Photon im UV-Bereich ein Elektron anregt in eine höheren Zustand überzugehen. Relaxiert diese nach einer gewissen Zeit $\tau$ in den Grundzustand und emittiert dabei ein Photon im sichtbaren Spektrum, wird der Vorgang Fluoreszenz genannt.

Das Aufspalten der Hyperfeinstruktur, die durch ein Magnetfeld erzeugt wird, ist für $\ce{^{85}_{}Rb}$ und $\ce{^{87}_{}Rb}$ in Abbildung XXX dargestellt. Für beide Isotope wird die Aufspaltung der Orbitale $5S_{1/2}$ und $5S_{3/2}$ betrachtet. Dabei spaltet das S-Orbital in zwei und das P-Orbital in vier Energieniveaus auf. Insgesamt sind vier Absorptionslinie zu beobachte (siehe Abbildung XXX). Der Grund dafür besteht darin, dass der Energieunterschied zwischen den Niveaus im P-Orbital im Gegensatz zum Unterschied im S-Orbital zu vernachlässigen ist. Bei der Betrachtung der Auspaltung der beiden Grundniveaus fällt zusätzlich auf, dass der Abstand bei $\ce{^{87}_{}Rb}$ um etwa Faktor zwei größer ausfällt, als bei $\ce{^{85}_{}Rb}$. Dies hat zur Folge, dass der Übergang 87b im Vergleich zu den anderen Übergängen die niedrigste Frequenz und 87a die höchste Frequenz aufweist. Dazwischen befinden sich, aufgrund der geringeren Aufspaltung, die Übergänge von $\ce{^{85}_{}Rb}$.  


\section{Justage und Durchführung}
In diesem Kapitel werden alle wesentlichen Schritte der Durchführung aufgelistet und die jeweilige Beobachtungen diskutiert. Es ist hier zu erwähnen, dass eine Speicherung eines Screenshots auf einem USB-Stick nicht möglich war, weshalb zur Dokumentation der Ergebnisse der Bildschirm des Oszilloskop mit einem Smartphone abphotographiert worden ist. 

\subsection{Stromminimierung}
Wie in Kapitel XXX beschrieben, erzeugt die Diode bei zu geringem Strom inkohärente Strahlung, ähnlich einer LED. Um dieses Zu beobachten wird eine IR-Indikatorkarte im Strahlengang des Lasers installiert. Mit einer UV-Kamera, die auf die IR-Karte fokussiert ist, kann nun der Laserspot über ein Bildschirm beobachtet werden. In diesem Teil der Durchführung muss eine UV-Kamera verwendet werden, da der Laserstrahl für das menschliche Auge nicht sichtbar ist, jedoch für die Kamera, da sie in diesem Wellenlängenbereich wesentlich sensitiver ist. Mit steigendem Strom $I$ steigt die Intensität linear an. Zu diesem Zeitpunkt weißt der Laserspot wohl definierte Grenze auf (siehe Abbildung XXX). Wird der 'Threshold' überschritten kommt es zur einer plötzlichen Intensitätsteigerung, die jedoch bei weiterem steigern des Strom erneut linear verläuft. An diesem Zeitpunkt ist der Laserspot und vor allem seine Grenzen als verschmiert oder diffus zu bezeichnen (siehe Abbildung XXX). 
Für eine optimale Ausrichtung des Lasers ist der Strom zu minimieren. Dies wird erreicht, in dem der Strom auf einen Wert kurz unter dem Threshold gestellt wird. Dann wird abwechselnd der Winkel des Gitters und die Länge des externen Resonators variiert bis der Laser kohärentes Licht emittiert. Erneut wird der Strom verringert, bis dieser sich unter dem Threshold befindet und die Prozedur erneut durchgeführt. Nach einigen Iterationsschritten kann der Strom nicht weiter minimiert werden. Um eine Aussage über den Threshold treffen zu können, wird der Strom mit Hilfe des Oszilloskops einmal unter- und einmal oberhalb des Oszilloskop gemessen. Die Ergebnisse lauten:
\begin{align}
	U_{unter}=\SI{5}{\milli\volt} \qquad \test{und} \qquad U_{über}=\SI{5}{\milli\volt}
\end{align}
Für die beiden Ergebnisse wird der Mittelwert und die Standardabweichung bestimmt. Somit ergibt sich ein Threshold von $I=5$.

\subsection{Lumineszenz von Rubidium}
Nach der optimalen Justage wird nun die IR-Indikatorkarte aus dem Strahlengang entfernt, sodass der Strahl durch die Rubidium-Zelle propagiert. Diese ist mit gasförmigen Rubidium gefüllt und aufgeheizt. Die UV-Kamera wird seitlich neben der Zelle platziert, sodass die Lumineszenz im weiteren Verlauf gut beobachtet werden kann. Die erst wird der Laserstrom erhöht. Als nächstes wird der Winkel des Gitters und die Länge des externen Resonators variiert, bis die richtige Wellenlänge (780) erreicht ist und somit zu Lumineszenz führt. Die Absorbtionslinie is in Abbildung XXX dargestellt. Es ist zusätzlich zu erwähnen, dass ein Filter und ein 50/50-Spiegel in dem Strahlengang platziert worden ist, um die Intensität des Lasers abzuschwächen.   

\subsection{Absortionsspektrum}
Das Ziel ist es nun alles vier Absorptionslinie auf dem Oszilloskop sichtbar zu machen. Hierfür wird der Piezokristall verwendet, dessen Volumenausdehnung proportional zur angelegten Spannung ist. Somit kann die Länge des externen Resonators kontinuierlich variiert werden, wodurch die Wellenlänge des Lasers verändert wird. Angeschlossen an einen Rampengenerator wird die Expansion und Kompression des Kristalls mehrfach pro Sekunde durchgeführt. Die Folge ist ein schnelles Flackern in der Rubidium-Zelle.  


% 2x2 Plot
% \begin{figure*}
%     \centering
%     \begin{subfigure}[b]{0.475\textwidth}
%         \centering
%         \includegraphics[width=\textwidth]{Abbildungen/Schaltung1.pdf}
%         \caption[]%
%         {{\small Schaltung 1.}}
%         \label{fig:Schaltung1}
%     \end{subfigure}
%     \hfill
%     \begin{subfigure}[b]{0.475\textwidth}
%         \centering
%         \includegraphics[width=\textwidth]{Abbildungen/Schaltung2.pdf}
%         \caption[]%
%         {{\small Schaltung 2.}}
%         \label{fig:Schaltung2}
%     \end{subfigure}
%     \vskip\baselineskip
%     \begin{subfigure}[b]{0.475\textwidth}
%         \centering
%         \includegraphics[width=\textwidth]{Abbildungen/Schaltung4.pdf}    % Zahlen vertauscht ... -.-
%         \caption[]%
%         {{\small Schaltung 3.}}
%         \label{fig:Schaltung3}
%     \end{subfigure}
%     \quad
%     \begin{subfigure}[b]{0.475\textwidth}
%         \centering
%         \includegraphics[width=\textwidth]{Abbildungen/Schaltung3.pdf}
%         \caption[]%
%         {{\small Schaltung 4.}}
%         \label{fig:Schaltung4}
%     \end{subfigure}
%     \caption[]
%     {Ersatzschaltbilder der verschiedenen Teilaufgaben.}
%     \label{fig:Schaltungen}
% \end{figure*}
