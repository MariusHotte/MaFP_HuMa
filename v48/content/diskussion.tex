\section{Diskussion}
\label{sec:Diskussion}
Die Hauptquelle der Unsicherheiten auf die Messergebnisse in Kapitel \ref{sec:Auswertung} sind auf die Ausgleichsrechnungen zurückzuführen. Besonders die Unsicherheit auf die Berechnung des Untergrunds hängt stark von den als Datengrundlage verwendeten Daten ab. Besonders fällt auf, das in beiden Messreihen die Ströme nach dem zweiten Maximum unter den vermuteten Untergrund fallen. Somit basiert der berechnete Untergrund nur auf Daten vor und nach dem ersten Maximum. 

Unter der Betrachtung der ersten Heizrate in Kapitel \ref{sec:hohe_heizrate} fällt auf, dass die Berechnung der Aktivierungsenergie unter den beiden Verfahren für den erste Peak übereinstimmende Ergebnisse liefert. Für den zweiten Peak besteht jedoch eine Abweichung von $\Delta W_{Peak2}= \SI{0.125+-0.045}{\percent}
$ zwischen den Verfahren. Der Grund dafür ist in den fehlenden Untergrunddaten nach dem zweiten Maximum zu vermuten. 

Bei der Auswertung der Messdaten, die mit der niedrigeren Heizrate aufgenommen wurden, fällt auf, dass die Aktivierungsenergie für das zweite Maximum kleiner ist als für das erste Maximum. Die Quelle dieses Fehlers ist erneut in den Untergrunddaten zu vermuten.

Der Vergleich der minimale Relaxationszeiten für die beiden Heizarten und den beiden Auswertungsverfahren zeigt, dass zu einen die Unsicherheiten sehr groß ausfallen und zum anderen zwischen der Ergebnisse der beiden Verfahren knapp eine Größenordnung Unterschied zu beobachten ist. Aus beiden Umständen folgt, dass die berechneten minimalen Relaxationszeiten nicht aussagekräftig sind. 