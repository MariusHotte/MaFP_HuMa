\section{Diskussion}
\label{sec:Diskussion}

Bei dem durchgeführten Versuch ist zwischen den Unsicherheiten der Messinstrumente und den mathematischen Unsicherheiten, die aus den Ausgleichsrechnungen hervorgehen, zu unterscheiden. Die Quelle der Messunsicherheiten besteht zum einen aus dem Amperemeter und zum anderen aus dem Temperaturmessgerät. Diese Unsicherheiten werden in der Auswertung als vernachlässigbar angesehen. Die mathematischen Unsicherheiten ergeben sich ausschließlich aus den durchgeführten Ausgleichsreichungen, sowie durch die anschließend durchgeführten Fehlerfortpflanzungen. Besonders die Unsicherheit auf die Berechnung des Untergrunds hängt stark von den als Datengrundlage verwendeten Messdaten ab. Besonders fällt auf, das in beiden Messreihen die Ströme nach dem zweiten Maximum unter den vermuteten Untergrund fallen. Somit basiert der berechnete Untergrund nur auf Daten vor und nach dem ersten Maximum. 

Unter der Betrachtung der ersten Heizrate in Kapitel \ref{sec:hohe_heizrate} fällt auf, dass die Berechnung der Aktivierungsenergie unter den beiden Verfahren für das erste Maximum nur eine Abweichung von $\Delta E = \SI{4+-10}{\percent}
$ liefert. Die Relaxationszeiten weisen im Gegensatz dazu eine Abweichung von $\Delta\tau = \SI{-705+-3298}{\percent}
$ auf. Aufgrund der kleinen Abweichung der Aktivierungsenergien ist zu schließen, dass der große Unterschied in den Relaxationszeiten nicht durch die Verwendung eines anderen Auswertungsverfahren hervorgerufen wird, sondern eine nicht berücksichtige Fehlerquelle.

Das im vorherigen Abschnitt beschriebene Phänomen ist ebenfalls bei der Auswertung der zweiten Heizrate mit einer Abweichung von $\Delta E = \SI{-10.5+-8.2}{\percent}
$ und $\Delta\tau = \SI{98.6+-4.6}{\percent}
$ zwischen den Verfahren zu beobachten.

Der Vergleich der Relaxationszeiten für die beiden Heizarten und den beiden Auswertungsverfahren zeigt, die Unsicherheiten sehr groß ausfallen. Aus diesem Umstand folgt, dass die berechneten Relaxationszeiten nicht aussagekräftig sind. 